\documentclass[12pt]{article}
\usepackage[utf8]{inputenc}
\usepackage[T1]{fontenc}
\usepackage{amsmath,amsfonts,amssymb}
\usepackage{graphicx}
\usepackage{a4wide}
\usepackage{hyperref}

\begin{document}

\paragraph{Title:} Online hand writing recognition

\paragraph{Abstract:} The task is classify text by stroke. The technical challenge originates from the variation of the handwriting style, the content is non-terogeneous 
. The content may be text, tables, diagrams, paints and others. Another difficulty is the use of such models on mobile devices. For use on mobile devices, the neural network must be lightweight.
The main goal of the work is to study existing approaches to solving this problem, to try a combined approach


\paragraph{Datasets:} 
\begin{enumerate}
\item CASIA-onDo~\cite{inbook}.
\item IAMonDo~\cite{inproceedings}.
\end{enumerate}

\paragraph{References:}
\begin{enumerate}
\item The formulation of the problem ~\cite{article}.
\item A baseline ~\cite{NIPS2008_66368270}
\item New result ~\cite{8978003}
\end{enumerate}

\paragraph{Basic solution:} Offline handwriting \href{https://github.com/suhaspillai/HandwritingRecognition-with-MultiDimensionalRecurrentNeuralNetworks}{recognition}

\paragraph{Authors:} Ilya Makarov, AIRI


\bibliographystyle{unsrt}
\bibliography{Name-theArt}
\end{document}