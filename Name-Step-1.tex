\documentclass[12pt]{article}
\usepackage[utf8]{inputenc}
\usepackage[T1]{fontenc}
\usepackage{amsmath,amsfonts,amssymb}
\usepackage{graphicx}
\usepackage{a4wide}
\title{Reconstructed abstract of the paper ``Generative or Discriminative? Getting the Best of Both Worlds''}
%\author{not specified, not necessary here}
\date{}
\begin{document}
\maketitle

\begin{abstract}
For many applications machine learning in the task predict the value y by design matrix X. The article
describes two methods for machine learning tasks: discriminative and generative approaches. The advantages and disadvantages of both approaches 
are considered. The main advantage of the generative approach is that it can work on unlabeled data. It is possible to combine both approaches into one model.
It's called discriminative training. There is only one correct way to train such a model. An example of applying the correct method for a task is considered
object recognition
\end{abstract}
\paragraph{Keywords:} Generative, Discriminative, Semi-Supervisedl learning, Unlabelled data, Combined model

\paragraph{Highlights:}
\begin{enumerate}
\item The benefit of the generative approach is that it can make use of unlabelled data
\item Discriminative training use to improve the predictive perfomance of generative models
\item Blending generative and discriminative approaches
\end{enumerate}

\section{Introduction}
Explain the reasons you chose the paper~\cite{homework_article}. The article actively discusses the Bayesian approach to machine learning.
This approach works well on small datasets under certain conditions. In particular, if we know some a priori knowledge about the data
One of the classic examples of this approach is the naive Bayesian classifier. Researching this approach can significantly improve the results of solving various machine learning problems on small datasets
%\begin{figure}
%\includegraphics[scale=0.35]{SVD_derint}
%\caption{A rigorous description of what the reader sees on the plot and the consequences of the shown result}
%\end{figure}

\bibliographystyle{unsrt}
\bibliography{Name-theArt}
\end{document}