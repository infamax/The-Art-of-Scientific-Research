\documentclass[12pt]{article}
\usepackage[utf8]{inputenc}
\usepackage[T1]{fontenc}
\usepackage{amsmath,amsfonts,amssymb}
\usepackage{graphicx}
\usepackage{a4wide}\title{Industrial project description: Recognition of Russian Sign Language}
%\author{not specified}
\date{}
\begin{document}
\maketitle

%\begin{abstract}
Answer the question to outline your project. Choose one of the roles: an {Expert} or an~\textbf{Analyst}. \textbf{Role:} Expert
%\end{abstract}
% \paragraph{Keywords:} The Art On Scientific Research, Abstract Reconstruction, Please Put Yours 


\section{Planning the industrial research project}
Before planning the research, the analyst and (\textbf{expert}) discuss the key issues. After the long dash~--- our remarks.

\begin{enumerate}
\item \textbf{Goal of the project:} The model translating from Russian Sign Language into Russian
\item \textbf{Applied problem solved in the project:} The result is to allow people who are deaf or hard of hearing to communicate with others using sign language.
Many Deaf smartphone users can fingerspell words faster than they can type on mobile keyboards. The model will help 
such people communicate more comfortably with others
\item \textbf{Description of historical measured data:} The data is a set of fullhd quality video files. 
Marked photographs with signs are also needed, what each sign means. 
\item \textbf{Quality criteria:} Levenshtein distance will be used as a quality metric. This metric shows the difference between two sequences.
\item \textbf{Project feasibility:} To begin with, as part of the PoC (Proof of concept), a model for recognizing individual signs 
will be implemented. If individual signs are successfully recognized, you can move on to recognizing entire sentences. 
\item \textbf{Conditions necessary for successful project implementation:} The model needs high resolution video. 
Poor video quality can dramatically reduce the quality of recognized signs. Important feature of Russian sign language is the presence of many dialects.
When creating a dataset, this must be taken into account and filled with representatives of different dialects. 
\item \textbf{Solution methods:} Yolo architecture may be used as baseline. Image Super-Resolution models should also be explored. Combined of Yolo architecture 
and Image Super-Resolution models can work this task. 
\end{enumerate}

\section{Research or development?}
In other words, novelty or technological advancement?

{Analyst:} What impact will the research have on the field of knowledge? How useful will it be?

{Expert:} (\textbf{How long will the model be used? What will replace it in the future?})

%\bibliographystyle{unsrt}
%\bibliography{Name-theArt}
\end{document}